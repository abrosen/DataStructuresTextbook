
\chapter{Introduction}

\section{What is a Data Structures Course}
Data Structures is all about defining the different ways we can organize data.


\section{Why This Book?}

This textbook is free.

It is both Java and Python.


\subsection{Where Does This Book Fit Into a Computer Science Curriculum }

Education in Computer Science is based around three core topics: translating the steps of solving a problem into a language a computer can understand, organizing data for solving problems, and techniques that can be used to solve problems. % reword
These courses typically covered in a university's introductory course, data structures course, and algorithms course respectively, although different universities decide exactly what content fits in which course.
Of course, there is are lot more concepts in computer science, from operating systems and low level programming,  to networks and how computers talk to each other. However, all these concepts rely on the knowledge gained in the core courses of programming, data structures, and algorithms.



This textbook is all about Data Structures, the middle section between learning how to program and the more advanced problem solving concepts we learn in Computer Science. 
Here, we focus on mastering the different ways to organize data, recognize the internal and performative differences between each structure, and learn to recognize the best (if there is one) for a given situation.


\subsection{What Are My Base Assumptions about the Reader?}

This textbook assumes that the student has taken a programming course that has covered the basics.
Namely: data types such as ints, doubles, booleans, and strings; if statements, for and while loops; and object orient programming.
This book is also suitable for the self taught programmer who has not learned much theoretical programming

\section{To The Instructor}

You'll note that this textbook lacks some of the features found in commercially available textbooks.  The biggest of these is slides.  



For the most part, slides are too static to help students understand how to code. 



Does the lack of varied exercises make cheating on assignments easier as semesters go on?  Yes, but that bridge was burned long ago.  
The cheating student can plagiarize from various websites or anonymously hire another to do their work for them.
However, the student who cheats isn't exactly clever and certainly hasn't been exposed to much game theory.  
They will often cheat from the same source.  

In addition, during the writing of this text, technologies such as GPTChat were released.  This hasn't so much burned the bridge as dropped napalm on the entire surrounding forest.  Newer technologies will then salt that earth. I recommend an open and honest dialogue with your students and at least 50\% of their grade being the result of evaluations and assessments you do in class.  This can range from proctored exams to flipping the classroom and giving students the chance to work  on homework in class, where they are much more likely to turn to you or their peers for help.

\subsection{How to Use}

%TODO creater tag for teacher edition
\section{To The Student}


\subsection{How to use}



%\section{Science and Art}
