\chapter{Analyzing Algorithms}
\label{chap:analysis}
Or we would look at the list, but we need to talk about Math.  Sorry for the bait-and-switch, but it will make sense shortly.

You don't need  much math to be a good programmer, but if you want to be an amazing programmer, you probably need math or very math adjacent skills.



\section{Cost}
Every function, operation, algorithm, or what have you that a computer performs has a \emph{cost}. 
In fact, there are always multiples costs;  we often just focus on the most important one or two costs.  

What is most important depends on context.
However, in the vast majority of cases, the most important cost to focus on is \textbf{time}.
When our program is eating away at our storage resources like a hungry child slurping up spaghetti, we can always go out and buy more memory/storage/RAM.
If our program requires a large amount of energy consumption, energy is readily available from a variety of sources: batteries, power plugs, internal combustion engines, the giant fusion reactor in the sky.


\subsubsection*{Measuring Cost}
When we measure cost, we need to do abstractly.  
When we measure the amount of time that an algorithm takes, we look at the number of operations that will be executed, not the overall elapsed time.

\subsection{Time}
A time cost is a measure of not just how long it takes a program to finish executing, bit also how the length of execution is affected by adding additional item.

Time is almost always \emph{the most important cost}. We cannot got out and buy another weeks worth of time.  You can't hand a bunch of money to the reaper and ask for a deferral. You can't buy another minute to spend with your mother\footnote{Call your mother.  She would love to hear from you.} 
Yes, processors get faster as technology marches on, but they get faster slowly and Moore's law ostensibly has its limits.
The only way to make our programs realistically run faster is to make them more efficient.  And \textbf{Big O notation} is the way we will be measuring that efficiency.


\subsection{Space}

For data structures, we will be measuring their space efficiency in terms of \textit{auxiliary} cost, in other words, how much extra space we need to use over the space used for the items itself.   To clarify, any data structure that contains $ n $ items of roughly the same size will use $ n \times \mathtt{sizeOf(item)}$ space at minimum, no matter what data structure we use.  This accounts for the 
Each data structures 

\subsection{Energy}
%\subsection{Other costs - Bandwidth}
While not something this book concerns itself with, some programmers need to be wary of the amount of energy  an algorithm consumes.  If energy is expensive and/or battery life needs to be conserved, then choosing an energy efficient algorithm might be a better choice, even if the time or space complexity is higher.  Some examples where energy use is a large concern  is Mobile Ad-Hoc Networks (MANETs) and battery powered cameras.




\section{What is Algorithm Analysis}
% Taken from 
% For Java https://runestone.academy/ns/books/published/javads/algorithm-analysis_what-is-algorithm-analysis.html?mode=browsing 
% For Python https://runestone.academy/ns/books/published/pythonds3/AlgorithmAnalysis/WhatIsAlgorithmAnalysis.html?mode=browsing

It is very common for beginning computer science students to compare their programs with one another. You may also have noticed that it is common for computer programs to look very similar, especially the simple ones. An interesting question often arises. When two programs solve the same problem but look different, is one program better than the other? 

In order to answer this question, we need to remember that there is an important difference between a program and the underlying algorithm that the program is representing. As you have learned, an algorithm is a generic, step-by-step list of instructions for solving a problem. It is a method for solving any instance of the problem so that given a particular input, the algorithm produces the desired result. A program, on the other hand, is an algorithm that has been encoded into some programming language. There may be many programs for the same algorithm, depending on the programmer and the programming language being used. 

To explore this difference further, consider the functions shown in Listing \ref{code:sumOfNJava1} and \ref{code:sumOfNPython1}. This function solves a familiar problem, computing the sum of the first n integers. The algorithm uses the idea of an accumulator variable that is initialized to 0. The solution then iterates through the $n$ integers, adding each to the accumulator
\begin{minted}[escapeinside=!!]{Java}
public class FindSum { !\label{code:sumOfNJava1}!
	public static long sumOfN(int n) {
		long theSum = 0;
		for (int i = 1; i <= n; i++) {
			theSum = theSum + i;
		}
		return theSum;
	}
	
	public static void main(String[] args) {
		System.out.println(sumOfN(10));
	}
}
\end{minted}

\begin{minted}[escapeinside=!!]{Python3}
def sum_of_n(n): !\label{code:sumOfNPython1}!
	the_sum = 0
	for i in range(1, n + 1):
		the_sum = the_sum + i

return the_sum

print(sum_of_n(10))
\end{minted}

Now look at the functions in Listing \ref{code:sumOfNJava2} and \ref{code:sumOfNPython2}. At first glance it may look strange, but upon further inspection you can see that this function is essentially doing the same thing as the previous one. The reason this is not obvious is poor coding. We did not use good identifier names to assist with readability, and we used an extra assignment statement that was not really necessary during the accumulation step.

\begin{minted}[escapeinside=!!]{Java}
public class FindSum2 { !\label{code:sumOfNJava2}!
	public static long foo(int tom) {
		long fred = 0;
		for (int nancy = 1; nancy <= tom; nancy++) {
			long joanne = nancy;
			fred = fred + joanne;
		}
		return fred;
	}
	
	public static void main(String[] args) {
		System.out.println(foo(10));
	}
}
\end{minted}

\begin{minted}[escapeinside=!!]{Python3}
def foo(tom): !\label{code:sumOfNPython2}!
	fred = 0
	for bill in range(1, tom + 1):
		barney = bill
		fred = fred + barney
	
	return fred

print(foo(10))
\end{minted}

%This block from the Java book
The question we raised earlier asked whether one method is better than another. The answer depends on your criteria. The method sumOfN is certainly better than the method foo if you are concerned with readability. In fact, you have probably seen many examples of this in your introductory programming course since one of the goals there is to help you write programs that are easy to read and easy to understand. In this course, however, we are also interested in characterizing the algorithm itself. (We certainly hope that you will continue to strive to write readable, understandable code.)

Algorithm analysis is concerned with comparing algorithms based upon the amount of computing resources that each algorithm uses. We want to be able to consider two algorithms and say that one is better than the other because it is more efficient in its use of those resources or perhaps because it simply uses fewer. From this perspective, the two methods above seem very similar. They both use essentially the same algorithm to solve the summation problem.

At this point, it is important to think more about what we really mean by computing resources. There are two different ways to look at this. One way is to consider the amount of space or memory an algorithm requires to solve the problem. The amount of space required by a problem solution is typically dictated by the problem instance itself. Every so often, however, there are algorithms that have very specific space requirements, and in those cases we will be very careful to explain the variations.

As an alternative to space requirements, we can analyze and compare algorithms based on the amount of time they require to execute. This measure is sometimes referred to as the execution time or running time of the algorithm. One way we can measure the execution time for the function sumOfN is to do a benchmark analysis. This means that we will track the actual time required for the program to compute its result. In Java, we can benchmark a function by noting the starting time and ending time within the system we are using. In the System module there is a method called nanoTime that will return the amount of time that the Java virtual machine has been running, in nanoseconds. By calling this function twice, at the beginning and at the end, and then computing the difference, we can get the number of seconds (fractions in most cases) for execution.

Listing 2.2.3 lets you enter the number you want to sum up to (n). It then invokes the sumOfN method 25 times and calculates the time required for each trial:

\begin{minted}[escapeinside=!!]{Java}
import java.util.Scanner;

public class Timing {
	public static long sumOfN(long n) {
		long theSum = 0;
		for (int i = 1; i <= n; i++) {
			theSum = theSum + i;
		}
		return theSum;
	}
	
	public static void main(String[] args) {
		
		Scanner input = new Scanner(System.in);
		System.out.print("Find sum from 1 to n: ");
		long n = input.nextInt();
		
		for (int trial = 0; trial < 25; trial++) {
			long startTime = System.nanoTime();
			long result = sumOfN(n);
			
			double elapsed = (System.nanoTime() - startTime) / 1.0E9;
			System.out.printf("Trial %d: Sum %d: time %.6f sec.%n",
			trial, result, elapsed);
		}
	}
}
\end{minted}

Listing 2.2.3 shows the original sumOfN function with the timing calls embedded before and after the summation. The function returns a tuple consisting of the result and the amount of time (in seconds) required for the calculation. Here is start and end of the output when we enter 10000 for the limit of the sum:

\begin{verbatim}
Trial 0: Sum 50005000: time 0.003886 sec.
Trial 1: Sum 50005000: time 0.004009 sec.
Trial 2: Sum 50005000: time 0.000186 sec.
Trial 3: Sum 50005000: time 0.000185 sec.
...
Trial 20: Sum 50005000: time 0.000125 sec.
Trial 21: Sum 50005000: time 0.000124 sec.
Trial 22: Sum 50005000: time 0.000125 sec.
Trial 23: Sum 50005000: time 0.000124 sec.
Trial 24: Sum 50005000: time 0.000124 sec.
\end{verbatim}


Why does the time go down from .003886 seconds to .000124? Because the Java Virtual machine and the computer hardware itself cache results, keeping them in memory for future access. We run the trial loop 25 times in order to give the cache time to stabilize.

We discover that the time is fairly consistent and it takes on average about 0.000125 seconds to execute that code. What if we run the function adding the first 100,000 integers? (Showing only the final five trials)

\begin{verbatim}
Trial 20: Sum 5000050000: time 0.001225 sec.
Trial 21: Sum 5000050000: time 0.001226 sec.
Trial 22: Sum 5000050000: time 0.001225 sec.
Trial 23: Sum 5000050000: time 0.001224 sec.
Trial 24: Sum 5000050000: time 0.001224 sec.
\end{verbatim}


Again, the time required for each run, although longer, is very consistent, averaging about 10 times more seconds. For $n = 1,000,000$  we get:


\begin{verbatim}
Trial 20: Sum 500000500000: time 0.012350 sec.
Trial 21: Sum 500000500000: time 0.012411 sec.
Trial 22: Sum 500000500000: time 0.012353 sec.
Trial 23: Sum 500000500000: time 0.012443 sec.
Trial 24: Sum 500000500000: time 0.012447 sec.
\end{verbatim}

In this case, the average again turns out to be about 10 times the previous experiment.

Now consider Listing 2.2.4, which shows a different means of solving the summation problem. This method, sumOfNImproved, takes advantage of a closed equation $\sum_{i=1}^{n} i = 1+2+3+\dots+(n-1)+n =  \frac {(n)(n+1)}{2}$
to compute the sum of the first n integers without iterating\footnote{This sequence of numbers is the \textbf{Triangular Number Series} and shows up a lot in analysis.}.
\begin{minted}[escapeinside=!!]{Java}
public static long sumOfNImproved(long n) {
	long theSum = n * (n + 1) / 2;
	return theSum;
}
\end{minted}

We then change the call in line 23 of Listing 2.2.3 to:
\begin{verbatim}
long result = sumOfNImproved(n);
\end{verbatim}

If we do the same benchmark measurement with this revised code, using five different values for n (10,000, 100,000, 1,000,000, 10,000,000, and 100,000,000), we get the following results from averaging the last five trials:
\begin{verbatim}
Sum 50005000:       time 0.0000088 sec.
Sum 5000050000:     time 0.0000092 sec.
Sum 500000500000:   time 0.0000082 sec.
Sum 50000005000000: time 0.0000078 sec.
\end{verbatim}

There are two important things to notice about this output. First, the times recorded above are shorter than any of the previous examples. Second, they are very consistent no matter what the value of n. It appears that sumOfNImproved is hardly impacted by the number of integers being added.


But what does this benchmark really tell us? Intuitively, we can see that the iterative solutions seem to be doing more work since some program steps are being repeated. This is likely the reason it is taking longer. Also, the time required for the iterative solution seems to increase as we increase the value of n. However, if we ran the same function on a different computer or used a different method language, we would likely get different results. It could take even longer to perform sumOfNImproved if the computer were older.

We need a better way to characterize these algorithms with respect to execution time. The benchmark technique computes the actual time to execute. It does not really provide us with a useful measurement because it is dependent on a particular machine, program, time of day, compiler, and programming language. Instead, we would like to have a characterization that is independent of the program or computer being used. This measure would then be useful for judging the algorithm alone and could be used to compare algorithms across implementations.



\section{Big O Notation}

\begin{itemize}
	\item What is big O
	
	\item  how to read it
	\item Aside about big omega and theta
	\item How wrong usage annoys mathematician
	\item refers to cost in general, but used for time usually
	\item  space complexity 
	\item Common runtimes
	\item runtimes we''ll focus on now
	\item runtimes we focus on later
\end{itemize}



\subsection{Space Complexity}

\section{Examples with Arrays}

\begin{itemize}
	
	\item Retrieval  - refer back to earlier chapter for address lookup 
	\item Replacement
	\item Linear Search
	\item Binary Search
\end{itemize}



\subsection{Selection Sort}



\subsection{Bubble Sort}
\subsection{Insertion Sort}
\subsection{Other Sorting Algorithms}


\section{The Formal Mathematics of Big O Notation}
\section{Other Notations}


\section{When To Ignore Costs}