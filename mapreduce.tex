\chapter{Map Reduce}


\section{Map}
The \texttt{map()} operation\footnote{It is mildly confusing that there is a \texttt{map} data structure and a \texttt{map()} operation, so I will be marking the \texttt{map()} operation with a function invocation.} is a powerful function that may require us to think  differently about the way we have approached programming so far.

The map operation takes in 2 arguments, a collection and a function to apply to every item in the collection




When we are writing functions , we are creating new verbs for our programming language to use . These verbs take in arguments, nouns that we may have declared or defined ourselves. But one thing that we May not have done yet is passing a function as an argument to another function.

This is not an uncommon operation in mathematics Example listed below 



The semantics of this in every programming language is different , but the concept is the same 



Why introduces here? Because a lot of common operations that can be done with map reduce involve using hash tables 
