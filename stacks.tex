\chapter{Stacks}
Our next data structure is the Stack.
The stack may seem unnecessary as a data structure after we introduce its features.  
After all, can't a list do all the things that a stack can do and more? 

Working with the limited operations of a allows us to approach problems with a different mindset.

\section{Stack Operations}

The stack operations are limited and simple. 


\begin{enumerate}
	\item[Push] Put an item on the top of the stack.
	\item[Pop]  Remove the item from the top of the stack and return it.  The item that was underneath the top of the stack becomes the new top.
	\item[Peek] Return the top of the stack, without removing it.
\end{enumerate}


That's it.  That's all there is.  It is refreshingly simple.
There will usually be additional functions, such as one to check if the stack is empty or a function to get the number of items stored in the stack, but \texttt{push}, \texttt{pop}, and \texttt{peek} are the important ones.




\section{Building a Stack}

We will be building a stack as a reference-based structure in this book.  This is so we can get a bit more practice with manipulating nodes.


%https://runestone.academy/ns/books/published/pythonds3/BasicDS/ImplementingaStackinPython.html
	



\section{Built-in Stacks}

\section{Solving Problems with A Stack}


\section{Mazes - Stacks and Backtracking}



%\section{Discrete Finite Automata}
