\chapter{Recursion}

\section{Introduction}

\section{Recursive Mathematics}

\subsection{Fibonacci}
As it turns out, while this technically works...it's pretty terrible.  In short, using recursion, I managed to accidentally\footnote{All right, I did this totally on purpose.} write an $ O(2^n) $, or exponential time,  algorithm.  This is very bad.  This means increasing $ n $ by one \emph{doubles} the runtime of out algorithm!


\section{Redoing Things With Recursion}
Many of the things we are about to see should not be actually  used and serve only as examples, like our \texttt{printThis} function

\subsection{Printing Recursively}
\subsection{Recursive Linear Search}


\section{Binary Search}



\subsection{Runtime Analysis}




\subsubsection{How to not be scared of logarithms }
You may have learned that logarithms are the inverse operation to exponentiation.


This is an utterly useless definition when programming.


A more way of thinking about logarithms is "how many times can I recursively split something?"
For example, $\log b x$ asks "how many times can I recursively split my $x$ items into $b$ seperate piles?"

A more concrete example: $\log_2 16 = 4$, not because $2^4 = 16$, but because a pile of 16 items can be split in half into two piles of 8, each pile of 8 can be split in half into two piles of 4, the 4's can be split into 2's, the 2's into 1's --- four splits total:

<picture>

\subsubsection{Back to it.}

\section{Recursive Backtracking}
Recursion really comes in handy when we are trying to solve complex puzzles.
One of the most famous examples of 

\subsection*{The Recursive Backtracking Algorithm}


\subsection{Mazes Again}



\subsection{The Eight Queens Puzzle}

\subsubsection{Brute Force Solution}

\subsubsection{Recursive Solution Outline}

\subsubsection{A Place Holder For Validity}

\subsubsection{Performing the Recursion}

\subsubsection{Checking just One condition}


\subsubsection{Checking all the Conditions}




\subsection{Additional Problems left to the Reader}

\subsubsection{Knight's Tour}

\subsubsection{Sudoku}




\section{Recursive Combinations}



\section{Recursion and Puzzles}



\section{Recursion and Art}

\section{Recursion and Nature}
