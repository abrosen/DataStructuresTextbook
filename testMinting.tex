\documentclass[10pt,a4paper]{article}
\usepackage[T1]{fontenc}

\usepackage{tcolorbox}
\tcbuselibrary{minted}
\begin{document}

\definecolor{bg}{HTML}{282828} 

%https://tex.stackexchange.com/questions/505313/add-captions-and-labels-to-custom-tcolorbox-listing


\newtcblisting[auto counter]{mycbox}[1]{%
	colback=red!5!white,colframe=red!75!black,fonttitle=\bfseries,
	title=Listing \thetcbcounter: #1
}
	
	
\newtcolorbox[auto counter,number within=section,
number freestyle={(Q/\noexpand\thesection/\noexpand\Alph{\tcbcounter})},
]{phbox}[2][]{%
	colback=yellow!15!white,colframe=blue!75!black,fonttitle=\bfseries,
	title=Question~\thetcbcounter: #2,#1
}



\newtcolorbox[auto counter,number within=section]{pycode}[2][]{%
	colback=yellow!15!white,colframe=blue!75!black,fonttitle=\bfseries,
	title= Listing \thetcbcounter: #2,#1
	minted language=Python3,
	minted style=colorful
}



\begin{phbox}[label={myfreestyle}]{Title with freestyle number}
	This box is automatically numbered with \ref{myfreestyle} on page
	\pageref{myfreestyle}. Inside the box, the \thetcbcounter\ can
	also be referenced by |\thetcbcounter|.
	The real counter name is \texttt{\tcbcounter}.
\end{phbox}	




\begin{listing}
	content...
\begin{minted}[style=colorful, frame=lines]{c}
int main() {
	printf("hello, world");
	return 0;
}
\end{minted}
\caption{foo}
\end{listing}


\begin{minted}[style=monokai,bgcolor=bg]{c}
int main() {
	printf("hello, world");
	return 0;
}
\end{minted}

\begin{minted}[style=friendly_grayscale]{c}
int main() {
	printf("hello, world");
	return 0;
}
\end{minted}

\begin{verbatim}
int main() {
	printf("hello, world");
	return 0;
}
\end{verbatim}


\begin{tcolorbox}
	This is a \textbf{tcolorbox}.
\end{tcolorbox}

\begin{tcblisting}{colback=red!5!white,colframe=red!75!black}
	This is a \LaTeX\ example which displays the text as source code
	and in compiled form.
\end{tcblisting}


\begin{tcblisting}{colback=yellow!5,colframe=yellow!50!black,listing only,
		title=This is source code in another language (XML), fonttitle=\bfseries,
		listing engine=minted,minted language=xml}
	<?xml version="1.0"?>
	<project name="Package tcolorbox" default="documentation" basedir=".">
	<description>
	Apache Ant build file (http://ant.apache.org/)
	</description>
	</project>
\end{tcblisting}

\begin{tcblisting}{listing engine=minted,minted style=trac,
	minted language=java,
	colback=red!5!white,colframe=red!75!black,listing only}
public class HelloWorld {
	// A `Hello World' in Java
	public static void main(String[] args) {
		System.out.println("Hello World!");
	}
}
\end{tcblisting}


\newtcolorbox[auto counter,number within=section]{pabox}[2][]{%
	colback=red!5!white,colframe=red!75!black,fonttitle=\bfseries,
	title=Examp.~\thetcbcounter: #2,#1}
	
\begin{pabox}[label={myautocounter}]{Title with number}
	This box is automatically numbered with \ref{myautocounter} on page
	\pageref{myautocounter}. Inside the box, the \thetcbcounter\ can
	also be referenced by |\thetcbcounter|.
	The real counter name is \texttt{\tcbcounter}.
\end{pabox}

\newtcolorbox[use counter from=pabox]{mybox}[2][]{%
	colback=blue!5!white,colframe=blue!75!black,fonttitle=\bfseries,
	title=Some Box \thetcbcounter: #2,#1}
\begin{mybox}[label={tacos}]{Title with continued number}
	This box is automatically numbered with \ref{tacos} on page
	\pageref{tacos}. Inside the box, the \thetcbcounter\ can
	also be referenced by |\thetcbcounter|.
	The real counter name is \texttt{\tcbcounter}.
\end{mybox}



\ref{tacos} 


\end{document}