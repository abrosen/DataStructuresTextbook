\chapter{Functions and How They Work}

This will be an extremely short chapter, but an important one.

\section{The Runtime Stack}

\section{Argument vs Parameter}
An argument is the actual value you pass in, the parameter is the variable that accepts it.

<Programming example?>

\subsection{Does anyone actually care?}

I cared enough to look it up, but I also had to look it up to double check that I'm correct.  
In a casual situation or talking with another programmer, I don't actually think anyone would care, but I would take care to get it correct for your assignments and exams, much like you would take care to avoid using ``ain't'' in a formal essay.

\section{Passing Arguments}

The vast majority of programming languages are \textit{pass by copy} with a huge honking asterisk.
\begin{itemize}
	\item Pass by copy means that when something is input as the argument to a function, the function gets a copy of the thing you are passing to it.
	\item The \textit{huge honking asterisk} is that you are almost always passing a \textit{reference} or \textit{pointer} to an object, not the object itself.  The reason for this is that if we had a super mega huge object, copying it would take up a super mega huge amount of time and memory.
\end{itemize}



\subsection{How it Works in Java}
In Java, we have two broad categories of data types: primitives and objects.

When you pass a primitive, such as an \texttt{int} or \texttt{double}, the value gets copied from where it is stored in memory and copied into the argument.


\subsection{How it works in Python}

