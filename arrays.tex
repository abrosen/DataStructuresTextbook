\chapter{The Array}

\section{Why Arrays}

\begin{itemize}
	\item because new language: 
	Since this is a data structures course, I assumed that students have had exposure to arrays or array like objects.
	This chapter goes into a bit of a deeper detail that may have been glossed over and Introduces the topic in the appropriate language if need be.
	
	In other words, I assume you know what an array is , but not necessarily how to use it in Java or Python\footnote{Although we use lists in python}.
	\item  because internal memory lookup
	\item Because we need to make sure internal knowledge is cohesive (eg arrays of objects are arrays of pointers/references)
\end{itemize}







\section{Java and Arrays}
The Array is a built in class in Java, but the syntax is a bit unique \footnote{Enough so that I constantly had to look up how to do it my first two years of undergraduate studies, so don't feel too bad if you have to do the same}

To create an array in Java we do:

\begin{minted}{Java}
	Type[] myArray = new Type[sizeOfArray]
\end{minted}


Here, every item in the array is of whatever \texttt{Type} we want, which could be a Class or primitive.   
Arrays can be whatever integer size we desire, but once set it cannot be changed.
This is because to create an array, the computer allocates a contiguous block of memory.
If we wanted to resize it, there is no guarantee that this chunk of memory won't have things directly before or after it, preventing us from safely extending its range.

\section{Python and Arrays}
\label{sec-python-arrays-lists}
Python doesn't really do arrays.
It instead uses Lists, as we'll see in Chapter \ref{chap-arraylist}.
\texttt{myNotArray\footnote{On styles:  Java convention is to use camel case for variable types (\texttt{myVariableName}), while python convention is to use underscores (\texttt{my\_variable\_name}).  I will be using the Java style camel-casing for variables throughout the book for consistency and because it is my preference.} = []} does not actually make an array like you assume it would in some other language.  Instead it makes A list (specifically an arraylist ) to contain these items.
This works exactly like an array in other languages, but you get some nifty operations that allow us to dynamically resize this array if we need it bigger or smaller.\footnote{We cover the specifics in Chapter \ref{arraylist}}



However, if you really want or need to use an array in python, you can.
There are two ways to accomplish this.The first way is the built in \texttt{array} package.    The python package \texttt{numpy} contains

Why would we want to use a 


\subsection{Cool Ways to Build an Array in Python}







\section{How an Array Works}


\subsection{Operations}

\subsubsection{Retrieving a item stored at an index}

\subsubsection{Setting the value of an index}

\subsection{Array Internals and the Memory Formula}

\section{Common Array Algorithms}

\subsection{Finding Values in an Array}
\subsubsection{Finding the Minimum}

\subsubsection{Finding the Average}


\subsection{Limitations}
Most frequent characters
Resizing

