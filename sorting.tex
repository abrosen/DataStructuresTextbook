
\chapter{Sorting}


\section{Quadratic-Time Algorithms}

\subsection{Bubble Sort}

\subsection{Selection Sort}

\subsection{Insertion Sort}


\section{Log-Linear Sorting Algorithms}
The most commonly used sorting algorithms take $ O(n \lg(n)) $ time.
This is the hard limit on runtime %TODO Cite or correct
\subsection{Tree Sort}
The tree sort is the simplest algorithm to we will cover. Performing Tree sort is a matter of three simple steps

\begin{enumerate}
	\item Create a tree.
	\item Load the items you want to sort into the tree.
	\item Perform an inorder traversal of the tree.
\end{enumerate}


The performance of this algorithm depends completely on the type of tree we create for this algorithm.  Using a self-balancing binary search tree, adding $ n $ items to the tree takes $ O(n\lg(n)) $ and an in order traversal takes $ O(n) $ steps, for a grand total of $ O(n) $ runtime.  Using a binary search tree that does not self balance means that there is a worst case scenario of $ O(n^{2}) $ for adding all the $ n $ items.

Using a tree also means we use extra space since all the data has to be moved into a tree, using $ O(n) $ space.
\subsection{Heap Sort}
You might expect that heapsort deserves the same treatment as treesort.
After all, a heap has the same structure as a tree and both are constructed to perform operations in $log n $ time.

\subsection{Heapify}

\subsection{Quick Sort}
\subsection{Merge Sort}



\section{Unique Sorting Algorithms}


\subsection{Shell Sort}

\subsection{Radix Sort}

\section{State of the Art Sorting Algorithms}

\subsection{Tim Sort}
\subsection{Quick Sort}
\section{But What if We Add More Computers: Parallelization and Distributed Algorithms}



\subsubsection{Parallel VS Distributed}

\subsubsection{Map Reduce }


\section{Further Reading}